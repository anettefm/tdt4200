\documentclass[12pt, a4paper]{article} %Definerer skriftstørrelse, arktrype( eks a0paper, ..., a6paper, letterpaper....), dokumenttype(article, report, book, letter, beamer(presentasjon).   
\usepackage[font=small, labelfont=bf]{caption}
\usepackage{graphicx}
\usepackage{pdfpages}
\usepackage{amsmath, amssymb} %  matematiske funskjoner og symboler. 
\usepackage{listings}
\usepackage{url} % urls in the bibliograpy
\usepackage[utf8]{inputenc} % 
\usepackage{float}  %
\usepackage{subcaption} % NOT compatible with the subfig package
\usepackage{authblk}
\usepackage{siunitx}
\usepackage[parfill]{parskip}
\usepackage{enumerate}

\usepackage[  
left=20mm,
 right=20mm,
 top=10mm,
 bottom=20mm]{geometry} % kan endre på marger 
%%%%%%%%%%%%%%%%%%%%%%%%%%%%%%%%%%%%%%%%%%%%%%%%%%%%%%%%%%%%%%%%
% Gode infosider:
% https://no.sharelatex.com/learn/Creating_a_document_in_LaTeX 
% https://no.sharelatex.com/learn/Page_size_and_margins
 
%%%%%%%%%%%%%%%%%%%%%%%%%%%%%%%%%%%%%%%%%%%%%%%%%%%%%%%%%%%%%%%%
% Det som inkluderes i \maketitle
\title{Øving3\\ TDT4200}
\author[1]{Anette Fossum Morken}
\date{}
%%%%%%%%%%%%%%%%%%%%%%%%%%%%%%%%%%%%%%%%%%%%%%%%%%%%%%%%%%%%%%%
\begin{document}
\maketitle
\section*{Problem 1, Debugging}
\subsection*{a}
Koden frigjør ikke igjen minnet når programmet er ferdig med å kjøre.

\subsection*{b}

\subsection*{c}

\begin{itemize}
\item Linje 10: Allokerer minnet for unsignet char når det som skal være det er char.
\item Linje 13: Gir ikke lastChar noen verdi.
\item Linje 17: for-løkken går fra 10 til 0 noe som gjør at den looper gjennom 11 elementer.
\end{itemize}
Andre feil ved koden:
\begin{itemize}
\item Når pekeren mem er opprettet peker den på et sted i minnet, men verdiene til dette området er ukjent, slik at når man setter inn i stringen og om den er kortere enn 10 elementer vil det som var der etter lengden til stringen fortsatt være der. (satte inn memset(mem, sizeof(char), 10); etter malloc).
\item Programmet gir skjekker ikke og gir ikke ut noen feilmelding om stringen er lengre enn 10 karakterer.
\item Når inputet er lengre en 10 elementer kopierer programmet de 10 første elementete i inputet og printer ut disse ti i motesatt rekkefølge.
\end{itemize}
\section*{Problem 2, optimalisering}

\subsection*{d}
Kjøretid på Vilje:

\subsection*{e}
Bildene mine er like.

\subsection*{f}
sum+=imageIn->data[...] gjøres $11512800 $ ganger for en farge når $size=2$ og $39081600 $ ganger når $size=8$.

\subsection*{g}
Endringer som er gjort:
\begin{itemize}
\item Når performNewIdeaIteration kjøres gjør den utregninger for alle tre fargene, ikke for en og en slik den var før.
\item for-løkkene er delt opp slik at randbetingelsene behandles separat.
\item I stedet for de to innerste for-løkkene, blir pikslene summert radvis og lagret i en vektor. Vektoren blir summert en hang per kollonne og det er når $senterY=0$ for resten av kollonen blir den summen av en rad lagt til og summen av raden som nå ikke er innenfor size trukket fra. 
\item Alle summeringer er satt opp slik at når det itereres gjennom minnet gjøres det slik at man ikke "hopper" frem og tilbake i minnet, men mest optimalt etter hvordan C allokerer ting i minnet. 
\end{itemize}

Det er også en optimalisering som jeg ikke har gjort, det er å gjøre endringer på performNewIdeaFinalization, denne kan gjøres mer effektiv, men det er utenfor min kompetanse  C.

\end{document}
