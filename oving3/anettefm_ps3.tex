\documentclass[12pt, a4paper]{article} %Definerer skriftstørrelse, arktrype( eks a0paper, ..., a6paper, letterpaper....), dokumenttype(article, report, book, letter, beamer(presentasjon).   
\usepackage[font=small, labelfont=bf]{caption}
\usepackage{graphicx}
\usepackage{pdfpages}
\usepackage{amsmath, amssymb} %  matematiske funskjoner og symboler. 
\usepackage{listings}
\usepackage{url} % urls in the bibliograpy
\usepackage[utf8]{inputenc} % 
\usepackage{float}  %
\usepackage{subcaption} % NOT compatible with the subfig package
\usepackage{authblk}
\usepackage{siunitx}
\usepackage[parfill]{parskip}
\usepackage{enumerate}

\usepackage[  
left=20mm,
 right=20mm,
 top=10mm,
 bottom=20mm]{geometry} % kan endre på marger 
%%%%%%%%%%%%%%%%%%%%%%%%%%%%%%%%%%%%%%%%%%%%%%%%%%%%%%%%%%%%%%%%
% Gode infosider:
% https://no.sharelatex.com/learn/Creating_a_document_in_LaTeX 
% https://no.sharelatex.com/learn/Page_size_and_margins
 
%%%%%%%%%%%%%%%%%%%%%%%%%%%%%%%%%%%%%%%%%%%%%%%%%%%%%%%%%%%%%%%%
% Det som inkluderes i \maketitle
\title{Øving3\\ TDT4200}
\author[1]{Anette Fossum Morken}
\date{}
%%%%%%%%%%%%%%%%%%%%%%%%%%%%%%%%%%%%%%%%%%%%%%%%%%%%%%%%%%%%%%%
\begin{document}
\maketitle
\section*{Problem 1, Debugging}
\subsection*{a}
Koden frigjør ikke igjen minnet når programmet er ferdig med å kjøre.

\subsection*{b}

\subsection*{c}
Feil funnet:
\begin{itemize}
\item Linje 10: Allokerer minnet for unsignet char når det som skal være det er char.
\item Linje 13: Gir ikke lastChar noen verdi.
\item Linje 17: for-løkken går fra 10 til 0 noe som gjør at den looper gjennom 11 elementer.
\end{itemize}
Andre feil ved koden:
\begin{itemize}
\item Når pekeren mem er opprettet peker den på et sted i minnet, men verdiene til dette området er ukjent, slik at når man setter inn i stringen og om den er kortere enn 10 elementer vil det som var der etter lengden til stringen fortsatt være der. (satte inn memset(mem, sizeof(char), 10); etter malloc).
\item Programmet gir skjekker ikke og gir ikke ut noen feilmelding om stringen er lengre enn 10 karakterer.
\item Når inputet er lengre en 10 elementer kopierer programmet de 10 første elementete i inputet og printer ut disse ti i motesatt rekkefølge.
\end{itemize}
\section*{Problem 2, optimalisering}
\subsection*{b}
kjøretider på datasal:\\
real: 0m2.215s \\
user: 0m0.409s \\
sys 0m0.032s\\
\subsection*{c}
Kjøretid på CMB: 3.34s
\subsection*{d}
Kjøretid på Vilje:\\
real :   0m0.980s\\
user:    0m0.800s\\
sys :    0m0.104s\\

Dette er betydelig bedre enn tidene på datasalen(tulipan). Dette skyldes at chachen på datasalen er 8Mb mens på vilje er den 20Mb. 

\subsection*{e}
Bildene er ikke like i følge cheker.c, men det er innenfor feilmarginen. Feilen skyldes at for å øke antall elementer på på L1 er det redusert fra dobbel presisjon til singel presisjong.

\subsection*{f}
Det hentes infomasjon på formen imageIn-$>$data[...] $9206640$ ganger for en farge når $size=2$ og $9187920  $ ganger når $size=8$. Grunnen til at det er gjøres færre ganger når $size=8$ er fordi at områdene der randbetingelsene gjelder er større (her legges det eller trekkes kun fra ikke begge deler slik det gjøres i midtdelen).


\subsection*{g}
Endringer som er gjort:
\begin{itemize}
\item Når performNewIdeaIteration kjøres gjør den utregninger for alle tre fargene, ikke for en og en slik den var før.
\item for-løkkene er delt opp slik at randbetingelsene behandles separat.
\item programmet legger først sammen alle pikslene size fra en piksel i x-retning for alle pikslene og lagrer denne i et midlertidig bilde. Deretter går den gjennom dette midlertidige mildet og summerer alle pikslene i y-retning i det midlertidige bildet for så å lagre dte i en nytt midlertidig bilde. samtidig som denne summen lagres i det nye midlertidige bildet finnes gjennomsnittet av summen til pikselen og lagres i der samme bildet som ble sendt inn. 
\item alle summeringer blir gjort ved at summen i pikselene når senterX=0 (når det summeres i x-retning) og senterY=0(når det sumeres i y-retning) blir regnet ut ved bruk av for-løkke. Resten blir regnet ut ved å legge til pikselen size bortenfor( til høyre eller under pikselen det regnes ut for) og rekke fra pikselen size+1 før (til venstre eller over pikselen det regnes ut for).
\item Når det regnes ut i y retning gjøres dette også radvis bortover. 
\item Det er lagt til en kode copyImage som kopierer alle verdiene fra et allerede konvertert bilde. Dette gjør at man slipper å konvertere bildet fra ppm til flyttall mer enn en gang.
\item Det brukes single presision i stedet for dobbel. 
\end{itemize}

Det er også en optimalisering som jeg ikke har gjort, det er å gjøre endringer på performNewIdeaFinalization, denne kan gjøres mer effektiv, men det er utenfor min kompetanse  C.

\end{document}
