\documentclass[12pt, a4paper]{article} %Definerer skriftstørrelse, arktrype( eks a0paper, ..., a6paper, letterpaper....), dokumenttype(article, report, book, letter, beamer(presentasjon).   
\usepackage[font=small, labelfont=bf]{caption}
\usepackage{graphicx}
\usepackage{pdfpages}
\usepackage{amsmath, amssymb} %  matematiske funskjoner og symboler. 
\usepackage{listings}
\usepackage{url} % urls in the bibliograpy
\usepackage[utf8]{inputenc} % 
\usepackage{float}  %
\usepackage{subcaption} % NOT compatible with the subfig package
\usepackage{authblk}
\usepackage{siunitx}
\usepackage[parfill]{parskip}


\usepackage[ 
total={250mm,297mm},  
left=20mm,
 right=20mm,
 top=10mm,
 bottom=20mm]{geometry} % kan endre på marger 

 
%%%%%%%%%%%%%%%%%%%%%%%%%%%%%%%%%%%%%%%%%%%%%%%%%%%%%%%%%%%%%%%%
% Det som inkluderes i \maketitle
\title{øving 1 \\ TDT4200}
\author[1]{Anette Fossum Morken}
\date{}
%%%%%%%%%%%%%%%%%%%%%%%%%%%%%%%%%%%%%%%%%%%%%%%%%%%%%%%%%%%%%%%
\begin{document}
\maketitle

\section*{1}
\subsection*{a}
Flynns taksanomi er en klassifisering av dataarkitektur. Den gir antall instruksjoner og antall data prosessoren kan strømme samtidig, foreksempel single instruction stream, single data stream(SISM) som er et klassisk von Neumann system. De forskjellige klassene kan fremstilles i en tabell:

  \begin{tabular}{|@{}c@{}| @{}c@{} |@{}c@{}|@{} c@{} |@{}c@{}| }
    \hline
     & \textbf{Single Instruction} & \textbf{multiple instruction} & \textbf{single program} & \textbf{multiple program} \\ \hline
    \textbf{single data} & SISD & MISD & &  \\ \hline
    \textbf{multiple data} & SIMD & MIMD & SPMD & MPMD \\
    \hline
  \end{tabular}


Hvordan 
hvor 
hvorfor i Flynns taxonomy




\end{document}
